
\documentclass[../main.tex]{subfiles}

\begin{document}

\chapter{Requirements}

This chapter identifies the requirements of the software which is implemented in the context of this thesis.
Overall, there are three overall goals which need to be achieved. 

\begin{enumerate}[label=\Roman*.]
    \item Data owners need to be able to share (and revoke) their log data with other users in the system.
	\item Log data needs to be confidential by means of end-to-end encryption.
    \item Data owners should not be allowed to forge log data at any time.
\end{enumerate}


Traditional requirements engineering methods usually differentiate between functional and non-functional requirements based on the views of different stakeholders.
Since this thesis operates within the security domain, the traditional requirements engineering process needs to be adapted: 
Precise assumptions about what to protect and against whom to protect are mandatory for a successful implementation and evaluation.
Thus, this chapter closely follows the terms and methodology proposed by \citeauthor{Fabian2010} \cite{Fabian2010}.

\citeauthor{Fabian2010} fundamentally differentiate between functional, non-functional and security requirements, which are extracted from functional, non-functional and security goals.
Goals are rather abstract and vague formulations of what the system should achieve. 
They are refined into more detailed requirements for two reasons.
First, verifiable requirements allow us to compare different approaches and to finally choose one solution from the solution space. 
Second, they also help to verify and evaluate if the implemented solution satisfies the overall expectations. \cite{Fabian2010}

While the overall goal I will be treated as a functional goal (subsection \ref{functional-requriements}), the goals II and III are understood as security goals (subsection \ref{security-requriements}).
Further non-functional goals will be identified in subsection \ref{non-functional-requriements}.
Within each subsection the identified goals will be refined into requirements.
This finally results in a list of verifiable and consistent system requirements for this thesis (subsection \ref{system-requriements}).

Please note, that the identification of these requirements was not a linear process as presented in this chapter.
Instead, multiple iterations of reasoning about possible solutions and considering their implications for the inverse transparency toolchain were necessary.
This process finally lead to the detailed decomposition of the requirements depicted in the following.


\newpage
\section{Functional requirements}\label{functional-requriements}
Functional requirements describe \enquote{what the system does} \cite[11]{Mylopoulos1992}.
In the context of this thesis the following functional goals are identified:
\begin{itemize}
    \item Data owners need to to share their log data with other users in the system.
    \item Data owners need to revoke the access of other users to their log data
\end{itemize}

They can be refined into more concrete requirements. 
Let $L_{ID}$ denote an access log. 
This log reflects the fact, that a data consumer has accessed some data of the user with identity $ID$.
Let $Access(L_{ID})$ denote the set of all users, which can access the log $L_{ID}$.
E.g. $Access(L_{ID}) =\{1,2,3\}$ reflects the situation, where user $1$, $2$ and $3$ have access to $L_{ID}$.
\begin{enumerate}
    \item [F1.] Once a log was created, only the data owner can access the content the log:
    \\$Access(L_{ID}) =\{ID\}$
    \item [F2.] The data owner can share a log with other users in the system. 
    After the sharing operation the defined set of users can access the log.
    Assume the log is shared to the set of users $S$. 
    The sharing can be described as: 
    \\$Access(L_{ID}) = Access(L_{ID}) \cup S$
    \item [F3.] The data owner of a log can revoke the access to the log.
    After revocation operation the defined set of users can not access the log anymore.
    Assume the log is revoked from the set of users $R$. 
    The revocation can be described as: 
    \\$Access(L_{ID}) = Access(L_{ID}) \setminus R$

\end{enumerate}

\section{Security requirements}\label{security-requriements}
As proposed by \citeauthor{Fabian2010}, this section determines the security goals of this thesis and refines them into security requirements.
Each security goals is initially classified into confidentiality, integrity or availability (CIA-triad \todo{quelle}). 
From the overall goals II and III we obtain:
\begin{itemize}
    \item Access logs are end-to-end encrypted (confidentiality and availability)
    \item Access logs are integrity protected (integrity)
\end{itemize}

In order to identify security requirements, each goal is annotated with additional information, a counter-stakeholder and specific circumstances.
The counter-stakeholder is an adversary attacking the system.
Satisfying a security goal can be seen as defending against this adversary.
The circumstances elaborate additional conditions which affect the security goal.

\begin{enumerate}
    \item [S1.] Access logs are end-to-end encrypted.
    \begin{itemize}
        \item Information: 
        \todo{define E2E encryption}
        The content of the logs can only be accessed by communicating entities (encryption endpoints).
        The receiver of a access log needs to be able to verify that the access log was really shared by the intended sender. 
        The initial encryption endpoints are the monitor-component, which creates the access log, and the data owner.
        Once the access log is shared, each user who is allowed to access the access log becomes an encryption endpoint.
        Once access to a log is revoked for a user, this user is no longer an encryption endpoint.
        \item Counter-stakeholder:
        \todo{Dolev-Yao attacker}
        This security goal is intended to defend against a Dolev-Yao attacker.
        It is assumed that the attacker is the whole system, e.g. the attacker carries all communication.
        This is an active attacker not following the intended protocol.
        The attacker can modify, delete, add or duplicate messages.
        However, it is assumed that the attacker can not break correctly implemented cryptographic primitives. 
        \item Circumstances: 
        The dynamic behavior in the system implicates a dynamically changing set of encryption endpoints.
    \end{itemize}
    \item [S2.] Access logs are integrity protected.
    \begin{itemize}
        \item Information: 
        Once a access log was created, nobody can modify the access logs and insert it into the system.
        \item Counter-stakeholder: 
        This security goal is intended to defend against a \textbf{malicious data owner}, who tries to manipulate access logs. 
        The adversary tries to share manipulated logs with other users.
        Thus, it is assumed to be an active attacker not following the intended protocol.
        Further it is assumed that the attacker can not break correctly implemented cryptographic primitives.
        \item Circumstances: 
        The monitor component needs to initially create the access log. 
        After this creation no user can modify the log. 
    \end{itemize}
\end{enumerate}

\section{Non-functional requirements}\label{non-functional-requriements}


\section{System requirements}\label{system-requriements}



\end{document}
