
\documentclass[../main.tex]{subfiles}

\begin{document}


\chapter{Requirements}

This chapter identifies the requirements of the software which is implemented in the context of this thesis.
Overall, there are three overall goals which need to be achieved. 

\begin{enumerate}[label=\Roman*.]
    \item Data owners need to be able to share (and revoke) their log data with other users in the system.
	\item Log data needs to be confidential by means of end-to-end encryption.
    \item Data owners should not be allowed to forge log data at any time.
\end{enumerate}


Traditional requirements engineering methods usually differentiate between functional and non-functional requirements based on the views of different stakeholders~\todo{sources}.
Since this thesis operates within the security domain, the traditional requirements engineering process needs to be adapted: 
Precise assumptions about what to protect and against whom to protect are mandatory for a successful implementation and evaluation.
Thus, this chapter closely follows the terms and methodology proposed by \citeauthor{Fabian2010} \cite{Fabian2010}.

\citeauthor{Fabian2010} fundamentally differentiate between functional, non-functional and security requirements, which are extracted from functional, non-functional and security goals.
Goals are rather abstract and vague formulations of what the system should achieve. 
They are refined into more detailed requirements for two reasons.
First, verifiable requirements allow us to compare different approaches and to finally choose one solution from the solution space. 
Second, they also help to verify and evaluate if the implemented solution satisfies the overall expectations. \cite{Fabian2010}

While the overall goal I will be treated as a functional goal (subsection \ref{functional-requriements}), the goals II and III are understood as security goals (subsection \ref{security-requriements}).
Further non-functional goals will be identified in subsection \ref{non-functional-requriements}.
Within each subsection the identified goals will be refined into requirements.
This finally results in a list of verifiable and consistent system requirements for this thesis (subsection \ref{system-requriements}).

Please note, that the identification of these requirements was not a linear process as presented in this chapter.
Instead, multiple iterations of reasoning about possible solutions and considering their implications for the inverse transparency toolchain were necessary.
This process finally lead to the detailed decomposition of the requirements depicted in the following.


\newpage
\section{Functional requirements}\label{functional-requriements}
Functional requirements describe \enquote{what the system does} \cite[11]{Mylopoulos1992}.
In the context of this thesis the following functional goals are identified:
\begin{itemize}
    \item Data owners need to to share their log data with other users in the system.
    \item Data owners need to revoke the access of other users to their log data
\end{itemize}

They can be refined into more concrete requirements. 
Let $L_{ID}$ denote an access log. 
This log reflects the fact, that a data consumer has accessed data of the user with identity $ID$.
Let $Access(L_{ID})$ denote the set of all users, which can access the log $L_{ID}$.
E.g. $Access(L_{ID}) =\{1,2,3\}$ reflects the situation, where user $1$, $2$ and $3$ have access to $L_{ID}$.
It is mandatory, that during the whole lifetime of a access log the data owner can read the log.
Thus, the following is required at all times: $ ID \in Access(L_{ID})$.

This notation allows a exact definition of the functional requirements:

\subsubsection{F1. Creating access logs}
Once a log was created, only the data owner can access the content the log:
\begin{displaymath}
    Access(L_{ID}) =\{ID\}
\end{displaymath}

\subsubsection{F2. Sharing access logs}
The data owner can share a log with other users in the system. 
Assume the log is shared to the set of users $S$.
After the sharing operation has finished, each user in $S$ can access the log.
The sharing can be described as: 
\begin{displaymath}
    Access(L_{ID}) = Access(L_{ID}) \cup S
\end{displaymath}

\subsubsection{F3. Revoking access logs}
The data owner of a log can revoke the access to the log.
Assume the log is revoked from the set of users $R$.
After revocation operation has finished, no user in $S$ can access the log anymore.
The revocation can be described as:
\begin{displaymath}
    Access(L_{ID}) = Access(L_{ID}) \setminus R
\end{displaymath}

\section{Security requirements}\label{security-requriements}
As proposed by \citeauthor{Fabian2010}, this section determines the security goals of this thesis and refines them into security requirements.
Each security goals is initially classified into confidentiality, integrity or availability (CIA-triad \todo{quelle}). 
From the overall goals II and III we obtain:
\begin{itemize}
    \item Access logs are end-to-end encrypted (confidentiality, integrity and availability)
    \item Access logs are integrity protected (integrity)
\end{itemize}

These security goals can be refined into security requirements by annotating them with with additional information, a counter-stakeholder and specific circumstances.
The counter-stakeholder is an adversary attacking the system.
Satisfying a security requirement should effectively defend against this adversary.
The specified circumstances elaborate additional conditions which affect the security goal.

\subsubsection{S1. Access logs are end to end encrypted}

\paragraph{Information}
In the context of the inverse transparency toolchain, the intended end-to-end encryption should ensure that only authorized users can read access logs.
Specifically, no server within the system can access the content of these logs and a malicious server can not insert faked logs or modify existing data.

\paragraph{Counter-stakeholder}
This security goal is intended to implement E2EE in order to defend against an active adversary (details in section ~\ref{end-to-end}).
In the context of the inverse transparency toolchain, the attacker is adapted by the following means. 
The untrusted network is the whole infrastructure of the toolchain, except trusted entities and the application the user interacts with.
This implies, that all frontend applications are trusted (e.g. the Clotilde UI and all integrated Monitor components).
This thesis implements metrics to defend against the following scenarios:
\begin{itemize}
    \item A malicious entity within the untrusted network observes the data and tries to read the content of access logs.
    \item A malicious entity within the untrusted network modifies or replaces traffic to insert malicious, faked or misformed access logs.
\end{itemize}
\todo{Examples}
\paragraph{Circumstances}
The content of the logs can only be accessed by authorized entities (encryption endpoints).
The receiver of a access log needs to be able to verify that the access log was actually shared by the intended sender. 
The initial encryption endpoints are the monitor-component, which creates the access log, and the data owner.
Once the access log is shared by the data owner with the set of users $S$, each user in $S$ becomes an encryption endpoint.
Once the access to a log is revoked for a user, this user is no longer an encryption endpoint.
Thus, the dynamic behavior in the system implicates a dynamically changing set of encryption endpoints.

Note, that modified or inserted data can only be detected within trusted entities.
The malicious action itself can not be prevented.
Rather, mitigation mechanisms need to ensure, that malicious actions are detected and handled by trusted entities.

\subsubsection{S2. Access logs are integrity protected}

\paragraph{Information}
This thesis implements a new feature, allowing a data owner to share logs with other users within the system.
However, this also introduces the risk of a data owner creating faked logs or modifying existing logs.
Note, that this integrity protection is different from the protection provided by E2EE.
The later aims to protect the encrypted access log.
The protection required by this security goals protects the decrypted access log.
Once a shared log is received, the receiver must first verify if the encrypted data was modified during transit (integrity protection of E2EE).
Later, once the log was decrypted, the receiver additionally needs to verify if the provided access log was modified by an unauthorized entity.
       
\paragraph{Counter-stakeholder}
This security goal is intended to defend against a malicious data owner, who tries to manipulate access logs. 
Specifically, the adversary performs the following steps:
\begin{itemize}
    \item The malicious data owner decrypts the access log as intended.
    \item The malicious data owner modifies the decrypted data.
    \item The malicious data owner finally shares the modified data with other users.
\end{itemize}
Thus, it is assumed to defend against an active attacker not following the intended protocol.
Further it is assumed, that the attacker does not have the computational power to to break cryptographic primitives without knowing the keys.

\paragraph{Circumstances}
An authorized monitor component initially create access logs. 
Upon creation, these access logs can only be accessed by the data owner (see goal F1).
The data owner might want to share these access logs.
Each time a user receives a shared access logs, the user needs to verify if the access log was created by a authorized monitor.

\section{Non-functional requirements}\label{non-functional-requriements}
Non-functional requirements can be characterized as \enquote{global requirements on its development or operational costs, performance, reliability, maintainability, portability, robustness and the like} \cite[11]{Mylopoulos1992}.
In contrast to the functional and security requirements, the non-functional requirements are not a direct result of the overall requirements of the thesis.
Thus, they are not strictly mandatory.
At the same time, however, they add additional value to potential solutions.
Consider two solutions $A$ and $B$, which both satisfy all functional and security requirements.
The solution $A$ should be preferred over a solution $B$, if $A$ resolves more non-functional requirements than $B$.

During multiple iterations of elaborating potential solutions, the list of non-functional goals has grown to the following stable collection:


\subsubsection{N1. Trusted Entities}
By definition, trusted entities are not part of untrusted network. 
Thus, the proposed E2EE (detail in section \ref{security-requriements}) does not defend against them.
This implies, that a compromised trusted server undermines the security of the whole toolchain.
To reduce the risk and attack surface, the toolchain should rely on a minimal amount of trusted entities.

\subsubsection{N2. Standardized cryptographic algorithms}
The current implemented frontend of the inverse transparency toolchain runs within the browser. 
The Web Cryptographic API \cite{WebCryptoApi2017} defines an interface to perform cryptographic operations within browsers:
\enquote{[It] simply exposes already existing and often heavily verified cryptographic functionality to Web application developers through a standardized interface.} ~\cite[959]{Halpin2014}

This allows to run cryptographic operations outside of JavaScript and enables secure and performant cryptography within the browser~\cite{Halpin2014}.
This API, however, is limited to specific primitives. 
Advanced encryption schemes developed within recent years are not yet defined and available in the API.
To ensure portability of the inverse transparency toolchain, the implementation of this thesis should thus only rely on primitives defined by the Web Cryptographic API.

\subsubsection{N3. Usability}
Usually a trade-off between Security and Usability needs to be resolved during system design~\todo{quelle}.
In the context of this thesis, however, the complexity of security is a key-driver to improve usability. 
By integrating the error-prone cryptography tasks (e.g. encryption or integrity checks) into the application, we can increase the usability and thus the acceptance of the toolchain.
At the same time, the security is improved because these tasks are performed within the toolchain and can be heavily tested.
However, this comes with the cost of increased design and implementation effort.

Furthermore, this approach is motivated by the principle of least effort, which states that humans will spent the least amount of work to get a task done~\cite{Levenson2018}.

\subsubsection{N4. Simplicity of design}
To increase maintainability of the toolchain, a simple design with a minimal amount of infrastructure and communication should be implemented.
This also requires a proper documentation of the implemented features.

\subsubsection{N5. Resoure utilization}
This is a performance criterion.
The implemented solution should avoid resource overhead in terms of storage and computation.
While this is directly related to the non-functional requirement N2, more aspects need to be considered here.
Duplicated storage of data should be avoided.
Specifically, an optimal solution does not store the same data encrypted with different keys in the database to prevent redundancy.


\section{System requirements}\label{system-requriements}
This section summarizes all identified requirements elaborated in this chapter.
They were derived from the overall goals of this thesis.
For details see the corresponding subsections above.

\subsubsection{Functional requirements}
\begin{itemize}
    \item [F1.] Creating access logs
    \item [F2.] Sharing access logs
    \item [F3.] Revoking access logs
\end{itemize}

\subsubsection{Security requirements}
\begin{itemize}
    \item [S1.] Access logs are end-to-end encrypted
    \item [S2.] Access logs are integrity protected
\end{itemize}

\subsubsection{Non-functional requirements}
\begin{itemize}
    \item [N1.] Trusted Entities
    \item [N2.] Standardized cryptographic algorithms
    \item [N3.] Usability
    \item [N4.] Simplicity of design
    \item [N5.] Resource utilization
\end{itemize}

\end{document}
