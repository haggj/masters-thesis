\documentclass[../main.tex]{subfiles}

\begin{document}

\chapter{Overview of potential solutions}

This chapter theoretically elaborates on different approaches to implementing the requirements defined in chapter \ref{requirements}.
The sketched protocols illustrate different encryption techniques.
Each section details the consequences of the technique to the requirements of this thesis.
The outcome of this chapter is a set of possible solution strategies.

The following table summarizes the results of this chapter.
It provides an overview for the reader.
A checkmark (\checkmark) means that a approach satisfies the corresponding requirement.
A missing checkmark indicates that the approach conflicts with the requirement.
Details and justifications can be found in the corresponding sections below.


\begin{table}[h]
    \centering
    \begin{tabular}{|l|l|c|c|c|c|c|c|c|c|c|c|}
    \hline
    Section                         & Approach                  & F1          & F2          & F3            & S1            & S2            & S3            & N1            & N2            & N3            & N4            \\ \hline
    \ref{sec:external-encryption}   & External encryption       & \checkmark  & \checkmark  &               &               & \checkmark    & \checkmark    & \checkmark    & \checkmark    &               & \checkmark    \\ \hline
    \ref{sec:mutual-encryption}     & Mutual encryption         & \checkmark  & \checkmark  & \checkmark    & \checkmark    & \checkmark    & \checkmark    & \checkmark    & \checkmark    & \checkmark    &               \\ \hline
    \ref{sec:hybrid-encryption}     & Hybrid encryption         & \checkmark  & \checkmark  & \checkmark    & \checkmark    & \checkmark    & \checkmark    & \checkmark    & \checkmark    & \checkmark    & $\sim$        \\ \hline
    \ref{sec:key-server}            & Key server                & \checkmark  & \checkmark  & \checkmark    &               & \checkmark    & \checkmark    &               & \checkmark    & \checkmark    & \checkmark    \\ \hline
    \ref{sec:attribute-encryption}  & ABE                       & \checkmark  & \checkmark  & \checkmark    &               & \checkmark    & \checkmark    &               &               & \checkmark    & \checkmark    \\ \hline
    \ref{sec:broadcast-identity}    & IBBE                      & \checkmark  & \checkmark  & \checkmark    &               & \checkmark    & \checkmark    &               &               & \checkmark    & \checkmark    \\ \hline
    \ref{sec:broadcast-identity}    & CBBE                      & \checkmark  & \checkmark  & \checkmark    & \checkmark    & \checkmark    & \checkmark    & \checkmark    &               & \checkmark    & \checkmark    \\ \hline
    \ref{sec:broadcast-proxy}       & PRE                       & \checkmark  & \checkmark  & \checkmark    &               & \checkmark    & \checkmark    &               &               & \checkmark    & \checkmark    \\ \hline
\end{tabular}
\end{table}

\section{External encryption}
\label{sec:external-encryption}
An intuitive way of implementing a protocol that allows the exchange of logs among users is to rely on external encryption tools.
Suppose that Alice wants to share a log with Bob.
Alice could send the log to Bob with an encrypted email.
Emails can be encrypted with standardized protocols, e.g. PGP\footnote{Defined by IETF in \href{https://www.rfc-editor.org/rfc/rfc4880}{RFC4800}.} or S/MIME\footnote{Defined by IETF in \href{https://www.rfc-editor.org/rfc/rfc8551.html}{RFC8551}.}.
Such encryption technologies allow Alice to confidentially share logs with others.
This approach, however, is limited and does not satisfy the identified requirements.
It is the responsibility of the user to correctly apply the protocol.
This is error-prone, it could lead to unintended security issues and it is not user-friendly.
Sharing the log with $n$ users requires Alice to send $n$ encrypted emails.
More importantly, this approach does not end-to-end encrypt logs because the server storing the logs still has access to the unencrypted data.
Since the Overseer server is not trusted, confidentiality is broken (security requirement S1).
Once an encrypted email is sent to the receiver, there is no way for Alice to revoke any access to the log.
She loses control over the encrypted log because Bob's email provider stores the cipher.
This is not in the control of the toolchain.
It implies, that Alice can not revoke the access to the log anymore (functional requirement F3).
Those limitations motivate the implementation of a more sophisticated protocol.

\section{Mutual encryption}
\label{sec:mutual-encryption}
The term mutual encryption refers to the idea that logs are encrypted mutually between users in the system.
If each user is assigned a key pair, a data owner can encrypt a log for each user separately.
The core idea is similar to the approach presented for external encryption (section \ref{sec:external-encryption}).
However, this approach integrates the encryption and decryption of logs into the toolchain.
To better understand this approach consider the following example.
A monitor component accesses data of Alice and therefore creates a log.
Since Alice needs to be able to access the log, the monitor encrypts the log under the public key of Alice and sends it to the server.
If Alice keeps her secret key private she is the only entity who can decrypt the log.
She can download the log, decrypt it, and also share it with Bob by re-encrypting it under his public key.
The re-encrypted data is then sent to the server where it is stored.
Bob can finally download and decrypt the shared log.

Alice can access the log because the monitor encrypted it with her public key.
She can also share logs with others.
The encrypted data is stored by the Overseer server.
It allows Alice to request the deletion of the cipher from the server.
This way Alice can revoke access to the log.
Thus, all functional requirements are fulfilled.
As long as all users have exclusive access to their secret key, this approach also ensures the confidentiality of the log (security requirement S1).
To defend against malicious data owners creating forged logs, the monitor component (e.g. the entity creating a log) needs to cryptographically sign the log before encryption.
If Alice shares the log, she does not share the plain log.
Rather, she encrypts the signed log.
Alice can not modify the existing log or create a completely new log because she is not able to compute a valid signature in the name of the monitor.
She could only succeed by knowing the secret key of the monitor.
This structure allows the receiver to verify if the log was created by a valid monitor component.
As a consequence, a forged log can be detected during decryption (security requirement S3).
Security requirement S2 can also be satisfied.
To achieve this, Alice does not only encrypt the signed log.
She additionally signs and includes the identity of the receiving user.
This allows receivers to verify if the log was intended for them.
Figure \ref{fig:mutual_encryption} shows the structure of a log before it is encrypted.
The actual access log is signed by the monitor.
The shared log is a data structure that contains the signed access log.
It additionally specifies the user with whom the log is shared (e.g. receiver).
The shared log itself is signed by the data owner.
The signed shared log is passed to the encryption algorithm.
After decryption, a receiver can validate if the sender intentionally shared the log.
The receiver can also verify if the nested access log was signed by the claimed monitor.

\begin{figure}[ht]
    \includegraphics[scale=0.12]{../img/04/mutual_encryption.jpg}
    \centering
    \caption{Structure of a shared log before encryption.}
    \label{fig:mutual_encryption}
\end{figure}

This approach relies on standardized asymmetric cryptography (e.g. RSA).
While this approach fulfills all functional and security requirements it suffers from performance problems.
Similar to external encryption, the whole log needs to be encrypted $n$ times if it is shared with $n$ users.
Much more problematic, however, is the application of asymmetric cryptography to application data.
Public-key cryptography relies on intense mathematical computations and should not be used to encrypt large amounts of data directly~\cite[340]{Eckert2018}.
Eckert justifies this with the fact that asymmetric encryption algorithms are muss less performant than symmetric algorithms.
Following her argumentation, asymmetric cryptography is therefore usually used to encrypt a symmetric key which finally encrypts the larger payload.
Thus, the implemented protocol should avoid the encryption of log data with asymmetric algorithms.

\section{Hybrid encryption}
\label{sec:hybrid-encryption}

The approach extends the ideas introduced in the previous section \ref{sec:mutual-encryption}.
The concept of hybrid encryption combines the advantages of asymmetric cryptography with the advantages of symmetric cryptography.
In a hybrid scheme, a symmetric key is encrypted with asymmetric algorithms.
The symmetric keys are short (e.g. 128 or 256 bit for AES).
Thus, the expensive asymmetric algorithms only encrypt and decrypt little data.
The larger payload is then encrypted with a fast symmetric scheme using the exchanged key.
The asymmetric scheme is used to transport the symmetric key.~\cite[340]{Eckert2018}

Consider the scenario where Alice wants to share data with Bob.
Alice initially generates a symmetric key $k$.
She uses $k$ to encrypt the payload with a symmetric scheme (e.g. AES).
She then encrypts $k$ under the public key of Bob.
Finally, she sends the encrypted key along with the encrypted payload.
To decrypt the payload, Bob first decrypts the symmetric key $k$ using his secret key.
In the final step, Bob can apply $k$ to decrypt the payload.

This concept can be applied to the context of this thesis.
An initially created log is encrypted for Alice using hybrid encryption.
This results in a ciphertext that consists of the encrypted symmetric key and the encrypted log.
The encrypted key can be decrypted only by Alice.
Alice can use the key to decrypt the log.
If she wants to share the log with Bob, she applies hybrid encryption again:
She generates a fresh symmetric key and encrypts it under Bob's public key.
The log is encrypted with the fresh key.
The encrypted key and the encrypted log are sent to the server.
This allows Bob to decrypt the log.
Additionally, Alice can specify multiple receivers by attaching multiple encrypted keys to the encrypted log -- one for each receiver.
This is visualized in figure \ref{fig:hybrid_encryption}. 
The key $k$ is used to encrypt the log in a symmetric scheme.
To make the key available to Bob and Charlie the key is encrypted for both of them using asymmetric encryption.
Alice can also revoke access to the log by re-encrypting the log. 
This requires a fresh symmetric key $k$.

\begin{figure}[ht]
    \includegraphics[scale=0.2]{../img/04/hybrid_encryption.jpg}
    \centering
    \caption{Structure of an encrypted log using hybrid encryption.}
    \label{fig:hybrid_encryption}
\end{figure}

The above description shows, that hybrid encryption techniques can be used to satisfy all functional requirements.
Moreover, only those can decrypt which were explicitly specified during encryption.
This ensures that only authorized users can decrypt (security requirement S1).
To adhere to security requirements S2 and S3 the techniques introduced in section \ref{sec:mutual-encryption} can be applied with a little modification.
Instead of a single receiver, the shared log needs to contain the list of authorized users.
The shared log is finally encrypted with the symmetric key.
This allows receivers to validate if the log was intended for them and to verify if the actual log was modified during transit.
Additionally, standardized algorithms exist for symmetric and asymmetric cryptography.
This allows the construction of a secure and portable crypto library.

While this approach improves the performance compared to section \ref{sec:mutual-encryption}, there is still a drawback.
If the log is encrypted for $n$ users, the symmetric key needs to be encrypted $n$ times.

\section{Key server}
\label{sec:key-server}

The development of cloud computing and centralized software architectures motivated the idea of key servers~\cite{Seitz2003}.
A key server stores cryptographic keys on behalf of the user.
If a user authenticates against the server and has the required permissions it can access a particular key.
This can be useful in environments where multiple users need to access the same key.
The creator and owner of a key can upload the key to the key server.
If others require access to the key, the owner can modify the access policy.

The concept of a key server can also be applied to share encrypted logs.
Once a monitor creates a log because it accesses sensitive data of Alice, the monitor creates a symmetric key and encrypts the log under this symmetric key.
The encrypted log is sent to the Overseer.
To decrypt the log, Alice requires access to this symmetric key.
Thus, the monitor needs to upload the generated key to the key server which allows Alice to download it.
If she wants to share the log with Bob, there is no need to re-encrypt any data in the Overseer.
Rather, she needs to tell the key server that Bob is allowed to access this key.
Alice can also revoke access to the file by telling the key server that a particular user is not allowed to access the key anymore.
However, if a malicious user stored the key on his local machine, the revocation does not have any effect.
The encrypted log in the Overseer should therefore be re-encrypted with a freshly generated key.
This new key is then also uploaded to the key server.

This example shows that all functional requirements can be satisfied when utilizing a key server.
It can be implemented using standardized symmetric encryption schemes (e.g. AES).
The techniques introduced in section \ref{sec:mutual-encryption} to satisfy security requirements S2 (the receiver can verify it is a valid encryption endpoint) and S3 (the data owner can not forge logs) can be applied again.
Security requirement S1, however, can not be met because the system suffers from key escrow (the administrator of the key server has access to all keys).
The confidentiality of all logs is broken because this allows him to potentially decrypt all logs.
The utilization of a key server also implies that we require an additional trusted component.
This increases the attack surface of the system and does not adhere to the non-functional requirement N1 (minimal number of trusted entities).

\section{Attribute-based encryption}
\label{sec:attribute-encryption}
Attribute-based encryption is an approach to provide access control within the domain of cryptography~\cite{Bethencourt2007}. 
Without attribute-based encryption, access control techniques need to be implemented by a server. 
Access will be granted if the requesting user is allowed to read/write the data (e.g. if the user works for a certain company). 
Thus, the server ensures that only authorized entities access data. 
Consider the case where this server is compromised: 
The attacker has unlimited access to the stored data because the access control techniques are no longer in place and the data itself is not encrypted.
Attribute-based encryption schemes shift the logic of access control into encryption and decryption algorithms. 
A distinction is made between a set of possible attributes (e.g. ${A,B,C}$) and a policy. 
Policies are logical expressions over attributes (e.g. $A \land B \land \neg C$). 
A secret key of a user contains certain attributes. 
During encryption, a user specifies a policy that is encoded into the ciphertext.
A ciphertext can only be decrypted with a secret key fulfilling that policy.
Within the domain of \textit{ABE}, one can differentiate between \textit{ciphertext-policy ABE} and \textit{key-policy ABE}. 
The above-described scenario, where the policy is encoded into the ciphertext and the key of the user is checked against this policy, is called \textit{ciphertext-policy ABE}. 
On the contrary, \textit{key-policy ABE} describes the scenario where the policy is encoded into the key of the user.
Thus, a user can only decrypt a ciphertext that is annotated with the corresponding attributes.~\cite{Bethencourt2007}

The notation of \textit{ciphertext-policy ABE} can be used directly to encrypt data for multiple receivers. 
During encryption, a policy is defined. 
All users which are equipped with attributes fulfilling this policy can decrypt data.
By encoding the name of the valid receivers into the policy, the encrypting user could specify the set of users who can decrypt.
If the access of a user needs to be revoked, the cipher is re-encrypted with a new set of receivers.
This ensures that all required functional requirements are met.
One can again equip the logs with cryptographic signatures (as described in section \ref{sec:mutual-encryption}).
Those techniques fulfill the security requirements S2 (the receiver can verify it is a valid encryption endpoint) and S3 (the data owner can not forge logs).

Attribute-based encryption usually relies on a trusted key generation center~\cite{Sahai2009}.
It computes and distributes cryptographic keys.
This implies key escrow and harms security requirement S1 because the trusted server can potentially decrypt all ciphers.
Additionally, each trusted entity increases the attack surface of the toolchain and conflicts with the non-functional requirement N1 (minimal number of trusted entities).
There are also schemes that avoid a centralized trust center by employing a decentralized architecture~\cite{Vaanchig2018}.
These systems, however, can not be integrated into the centralized architecture of the existing toolchain.
A lot of papers are published in the field of attribute-based encryption.
However, no cryptographic standard exists conflicting with the non-functional requirement N2.
It is not included in the Web Cryptographic API~\cite{WebCryptoApi2017}.

\section{Broadcast encryption}
\label{sec:broadcast-encryption}

The notation of broadcast encryption (\textit{BE}) was first introduced by~\citeauthor{fiat1993broadcast}~\cite{fiat1993broadcast}. 
Since then many different approaches to broadcast encryption schemes were proposed. 
All of them solve the following challenge: 
How to broadcast encrypted data while only an explicitly defined set of users can decrypt the data?
This section introduces different approaches to realize broadcast encryption.


\subsection{Identity-based broadcast encryption} 
\label{sec:broadcast-identity}

Identity-based broadcast encryption schemes (\textit{IBBE}) are the result of merging identity-based encryption (\textit{IBE}) with broadcast encryption techniques (\textit{BE})~\cite{Sakai2007}.
\textit{IBE} was initially taken into consideration by Shamir in 1985~\cite{shamir1985}.
It is public-key cryptography where the public key of a user is a unique string, e.g. its email address. 
Secret keys are derived from this unique string via a dedicated derivation function. 
To compute secret keys, identity-based encryption schemes require a trusted third party that computes and distributes secret keys for each user.
Otherwise, everyone could compute the secret key of arbitrary users resulting in an inherent insecure public-key cryptography scheme.
Identity-based cryptography was later combined with broadcast encryption resulting in \textit{IBBE}~\cite{Sakai2007}.
In these systems, a message can be encrypted for a dedicated set of users, where each user is identified by a unique string.
The encryption algorithm requires a plaintext and a set of identities~\cite{shamir1985}.
Decryption requires the secret key of one of the users specified during encryption.

Suppose that Alice wants to share encrypted data with Bob.
Once Alice and Bob received their secret keys from the trust center, Alice can encrypt the log and specify Bob as a valid receiver.
She can then upload the cipher.
Bob can download and decrypt it using his secret key.
If Alice wants to revoke access of Bob, she needs to re-encrypt the cipher and upload it again.
Thus, all three functional requirements can be fulfilled.
The techniques from section \ref{sec:mutual-encryption} can be applied again.
This satisfies the security requirements S2 (the receiver can verify it is a valid encryption endpoint) and S3 (the data owner can not forge logs).

\textit{IBBE} schemes suffer from key escrow because a trusted party computes and distributes secret keys.
This breaks the confidentiality of the encrypted logs (security requirement S2).
The additional trusted entity also increases the attack surface breaking non-functional requirement N1 (minimal number of trusted entities).
A lot of papers are published in the field of \textit{IBE}.
However, no cryptographic standard exists for \textit{IBE} or \textit{IBBE}.
It is not included in the Web Cryptographic API~\cite{WebCryptoApi2017}.
This conflicts with the non-functional requirement N3.

\subsection{Certificate-based broadcast encryption}
\label{sec:broadcast-certificate}

Certificate-based encryption (\textit{CBE}) was introduced by \citeauthor{Gentry2003} in 2003~\cite{Gentry2003}. 
Later, this idea was adopted with broadcast encryption by allowing multiple receivers~\cite{Li2018, Fan2013}.
The construction of \textit{CBE} schemes is motivated by the following observations~\cite{Gentry2003}:

\begin{itemize}
    \item Classical public key encryption (\textit{PKE}) schemes suffer from the certificate revocation problem (detail in \cref{terms}).
    \item Identity-based encryption schemes suffer from key escrow (details in \cref{IBBE}).
    However, since they make use of implicit certification (details in \cref{IBBE}), \textit{IBE} schemes do not suffer from the certificate revocation problem.
    \item Combining both techniques -- \textit{PKE} and \textit{IBE} -- yields the new construction \textit{CBE}. 
    It neither suffers from key escrow nor from the certificate revocation problem.
\end{itemize}

Consider the scenario where Alice wants to send encrypted data to Bob. 
According to~\cite{Gentry2003}, \textit{CBE} schemes function as followed:

\begin{enumerate}
    \item Alice generates a key pair with a \textit{PKE} key generation algorithm. She keeps the secret key secret. 
    Not even the trust center is allowed to know this secret key.
    \item Alice requests a certificate for her public key. 
    This certificate is created by the trust center by executing the \textit{IBE} key generation algorithm. 
    Thus, this certificate can technically also be seen as the second secret key of Alice, which the trust center knows.
    \item Bob also executes steps 1 and 2.
    \item To encrypt data for Bob, Alice uses Bobs public key and encrypts the message twice: Once, with the \textit{IBE} encryption algorithm and once with the \textit{PKE} encryption algorithm. The doubly encrypted message is then sent to Bob. In particular, there is no need for Alice to check if Bob has a valid certificate. This avoids the certificate revocation problem.
    \item Upon receiving the encrypted message, Bob needs to double decrypt the cipher. 
    Once, with the \textit{IBE} decryption algorithm and once with the \textit{PKE} decryption algorithm. 
    This enforces the following.
    First, only Bob can apply the \textit{PKE} decryption, because only he knows the required secret key. 
    Second, Bob needs a valid certificate because otherwise, he could not apply the \textit{IBE} decryption. 
    In particular, the first fact avoids key escrow (because the trust center does not know the secret key of Bob) and the second fact avoids the certificate revocation problem (because Bob can only decrypt with a valid certificate).
\end{enumerate}
This idea was later adopted to broadcast encryption yielding a certificate-based broadcast encryption scheme (\textit{CBBE})~\cite{Li2018}.
Although this construction relies on a key generation center, it does not suffer from key escrow.
Once keys are established, a user can encrypt data for a dedicated set of users.
Only these users can decrypt the cipher.
In particular, the key generation center can not decrypt the cipher because it does not have access to the PKE secret key of a user.
Besides the functional requirements, \textit{CBBE} also satisfies the security requirements.
The distinction of shared log and signed log introduced in section \ref{sec:mutual-encryption} can be applied again.
Thus, all receivers can validate that they are intended decryption endpoints (security requirement S2) and that the access log was created by the claimed monitor (security requirement S3).
The notion of E2EE is also fulfilled because the trusted server can not decrypt the logs.

Unfortunately, no reference implementations of \textit{CBBE} exist. 
This fact is a major drawback in the context of this thesis.
These cryptographic systems are currently the subject of research.
There are no ongoing standardization processes to analyze their security.
As a result, the non-functional requirement N2 (standardized cryptographic algorithms) can not be met.
Using those schemes in a practical implementation is avoided in this thesis because the missing standards introduce an unpredictable risk.

\subsection{Proxy re-encryption based broadcast encryption}
\label{sec:broadcast-proxy}
Broadcast encryption schemes can be implemented with a technique named proxy re-encryption (\textit{PRE}).
Proxy re-encryption systems are cryptographic schemes, in which a third party (a.k.a. a \textit{proxy}) converts a ciphertext, which was originally intended for Alice, to a new ciphertext, which is intended for Bob. 
For this conversion, the proxy requires a re-encryption key. 
Further, the cryptographic system requires that the proxy does not obtain knowledge about the plaintext during re-encryption.~\cite{Chen2018}

Similar to the example given in~\cite{Chen2018}, consider a user Alice who stores encrypted data $E_A$ in the cloud.
$E_A$ can only be decrypted by Alice.
Suppose that the data center is also the re-encryption proxy.
To delegate this data to Bob, Alice creates a re-encryption key $rk_{Alice,Bob}$. 
This key is transferred to the proxy.
It allows the proxy to re-encrypt $E_A$ to $E_B$.
Bob can then download and only he can decrypt $E_B$.
Alice can revoke access to a log by instructing the server to delete the cipher for a dedicated user.

The proxy needs to be semi-trusted because with the knowledge of $rk_{Alice,Bob}$ it can re-encrypt all ciphers from Alice to Bob~\cite{Chen2018}.
This has important consequences for the security of the protocol implemented in this thesis.
Once Alice defined Bob as a valid receiver for a single log, the server could re-encrypt all ciphers of Alice for Bob.
This, however, affects the confidentiality of encrypted logs.
Not only authorized users might have access.
Users might collude with the server (which re-encrypts logs on behalf of the user) to obtain decrypted logs.
Overall, a proxy re-encryption based scheme does not adhere security requirement S1 (confidentiality of logs).
Notice that no cryptographic standard exists for proxy re-encryption based cryptography.
It is not included in the Web Cryptographic API~\cite{WebCryptoApi2017}.
This conflicts with the non-functional requirement N3.

\section{Further investigations}

Many modern communication applications rely on E2EE. 
This section investigates approaches of the services Zoom, CloudSeal and the instant messaging services Whatsapp and Signal.
It details the encryption techniques used by those services.
The goal of this section is to demonstrate what encryption techniques are practically used in modern E2EE applications.

All following techniques have in common that they do not rely on a PKI.
This restriction, however, is not necessary for this thesis because the inverse transparency toolchain is intended to be distributed in an enterprise context.
This simplifies the design of the final protocol.

\subsection{Zoom}
Zoom provides a video conference platform\footnote{\url{https://zoom.us/}}. 
It has integrated E2EE in their application to protect the communication among participants.
The enterprise claims that even the Zoom service provider can not decrypt the exchanged data.~\cite{Blum2020}

Once a meeting is started, the initiator of the meeting generates a meeting key.
This secret is shared among all users participating in the meeting.
Thus, the key needs to be transported securely to all users joining the meeting.
In particular, this requires the joining users to be online.
All communication is encrypted using the shared meeting key and a symmetric encryption schemes (AES-GCM).
~\cite{Isobe2021}

The protocol of Zoom is similar to the hybrid encryption described in section \ref{sec:hybrid-encryption}.
However, it can not be applied in the context of this thesis because all participating users must be online.
This assumption does not hold in the context of the inverse transparency toolchain.
Specifically, a monitor needs to be able to encrypt a log for the data owner even the data owner is offline.
The approach of Zoom shows, however, that hybrid encryption is practically used to broadcast encrypted data to multiple receivers.

\subsection{CloudSeal}
CloudSeal~\cite{Xiong2012} is a scheme to securely share and distribute files over cloud-based storages (e.g. AmazonS3).
Their E2EE relies on proxy re-encryption based encryption (see section~\ref{sec:broadcast-proxy} for details).
CloudSeal can currently not be provided as a web application because it relies on advanced cryptographic algorithms which are not included in the Wep Cryptographic API.
Thus, their clients are required to run on native operating systems.~\cite{Xiong2012}

As highlighted in section~\ref{sec:broadcast-proxy}, the construction of proxy re-encryption based schemes violates the identified requirements of this thesis.
CloudSeal is exemplarily listed here because it shows that broadcast encryption schemes are practically implemented to achieve E2EE.
It is not an option for this thesis because of the limited algorithms provided in the Wep Cryptographic API.

\subsection{Instant messaging services}
Whatsapp\footnote{\url{https://www.whatsapp.com/}} and Signal\footnote{\url{https://signal.org/}} are two popular instant messaging services.
They claim to implement E2EE and rely both on the Signal protocol.
The Signal protocol is a combination of the X3DH protocol~\cite{Marlinspike2016} and the Double Ratched protocol~\cite{Perrin2016}.
The X3DH protocol establishes a shared secret between two authenticated users.
This effectively allows the initialization of a secure channel between the users.
The shared secret is used in the Double Ratched protocol to exchange encrypted messages.~\cite{Marlinspike2016, Perrin2016}

The architecture of the Signal protocol establishes sessions between users.
Encrypted data can be exchange within those session.
Encryption and decryption always requires the initially established shared secret.
If a hardware device is available (e.g. a smartphone with a secure storage element), this shared secret can be stored securely. 
When encrypting data for a set of users (e.g. within a group chat) the plaintext is encrypted for each user separately.~\cite{Marlinspike2014}

Unfortunately, the concept of sessions relying on an initially exchanged secret key does not fit the architecture of the toolchain.
The Clotilde frontend allows users to login in a web application.
Within this application the encrypted data is downloaded and decrypted.
A session (as used in the Signal protocol) requires, however, that the exchange shared key is available for the user during decryption.
Since the user can login from arbitrary devices and browsers, the session key needs to be transported among those devices.
Either the user handles this manually (this conflicts with usability) or a trusted server stores the session key (this conflicts with the notion of E2EE).
In both ways the requirements of this thesis can not be met.
A possible solution to fix this is the implementation and distribution of the Clotilde frontend as an app.
This way, a secure session could be established among users and the shared secret could be securely stored in dedicated secure hardware storages which are not available in browsers.

\end{document}
